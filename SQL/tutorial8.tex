\documentclass{article}
\title{Tutorial 08: An Introduction to SQL Using Oracle's SQL*PLUS}
\author{
	Jin, Ziyang\\
	\texttt{\# 34893140}\\
	\texttt{f4a0b}
	\and
	Kim, Joon Hyung\\
	\texttt{\# 35183128}\\
	\texttt{l1m8}
}

\newcommand{\anyvar}{\underline{\hspace{0.8em}}}
\newcommand{\RA}{\textbf{RA}}
\newcommand{\DL}{\textbf{Datalog}}

\begin{document}
	\maketitle

\noindent Create SQL queries to answer each of the following questions:

\begin{enumerate}
\item Show all columns and rows in the \texttt{authors} table.
	\begin{verbatim}
	SELECT *
	FROM   authors;
	\end{verbatim}

\item Display the schema of the \texttt{titles} table.
	\begin{verbatim}
	describe titles
	\end{verbatim}
	
\item Display the first and last names of all authors, and make the column headings (labels) at the top ``FirstName" and ``LastName" (with no blanks). Use column aliases for this.
	\begin{verbatim}
	SELECT au_fname AS "FirstName", au_lnane AS "LastName"
	FROM   authors;
	\end{verbatim}

\item Which authors live in Walnut Creek? Show all columns of the table in your result.
	\begin{verbatim}
	SELECT *
	FROM   authors
	WHERE  city=`Walnut Creek';
	\end{verbatim}
	
\item Determine the orders that are incomplete. In particular, list the \texttt{title\_id} and the number of titles that still have to be shipped before the order is complete. Show only those rows where not as many titles have been shipped as have been ordered. Use the \texttt{salesdetails} table for this.
	\begin{verbatim}
	SELECT title_id, qty_ordered-qty_shipped
	FROM   salesdetails
	WHERE  qty_ordered > qty_shipped;
	\end{verbatim}

\item Consider the \texttt{editors} table. List all editors (first and last names) who do not have a boss.
	\begin{verbatim}
	SELECT ed_fname, ed_lname
	FROM   editors
	WHERE  ed_boss IS NULL;
	\end{verbatim}

\item Which editors do \textit{not} have 993-86-0420 as a boss? List the last name of all such editors and their bosses. Hint: This may not be quite as easy as it looks because of presence of NULLs.
	\begin{verbatim}
	SELECT ed_lname, ed_boss
	FROM   editors
	WHERE  ed_boss IS NULL OR ed_boss <> `993-86-0420';
	\end{verbatim}

\item Which business books cost between \$20 and \$30, inclusive? List the title and price of each such book.
	\begin{verbatim}
	SELECT title, price
	FROM   titles
	WHERE  type=`business' AND price >= 20 AND price <= 30;
	\end{verbatim}

\item List the last names of all authors who have the letter 'y' or 'Y' in their last name.
	\begin{verbatim}
	SELECT au_lname
	FROM   authors
	WHERE  au_lname LIKE `%y%' OR au_lname LIKE `%Y%';
	\end{verbatim}

\item List all titles followed by the name of their publisher. Order them alphabetically by title.
	\begin{verbatim}
	SELECT title, pub_name
	FROM   titles, publishers
	WHERE  titles.pub_id = publishers.pub_id
	ORDER BY title;
	\end{verbatim}

\item List the first and last names of all authors of the book called ``Secrets of Silicon Valley".
	\begin{verbatim}
	SELECT au_fname, au_lname
	FROM   authors
	WHERE  au_id IN (SELECT au_id
	                 FROM titleauthors, titles
	                 WHERE title = `Secrets of Silicon Valley' AND
	                       titleauthors.title_id = titles.title_id);
	\end{verbatim}

\item List the titles of books that have more than one author. Only list the title once for each book.
	\begin{verbatim}
	SELECT title
	FROM   titles
	WHERE  title_id IN (SELECT A.title_id
	                    FROM titleauthors A, titleauthors B
	                    WHERE A.title_id = B.title_id AND A.au_id <> B.au_id);
	\end{verbatim}

\item Create an SQL query of your own choice that is substantially different from the queries above. Your query should join 3 tables and explicitly sort the results in a meaningful way (e.g., ascending or descending order by one of the attributes). Your output must display at least 3 rows in the result set. In addition to listing your query, copy-and-paste the answer to your SQL query (i.e., the results) in your handin deliverables.

	
\end{enumerate}



\end{document}