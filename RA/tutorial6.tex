\documentclass{article}
\usepackage{sectsty}
\title{Tutorial 6: Basic Relational Algebra}
\author{
	Jin, Ziyang\\
	\texttt{\# 34893140}\\
	\texttt{f4a0b}
	\and
	Kim, Joon Hyung\\
	\texttt{\# 35183128}\\
	\texttt{l1m8}
}
\date{February 2018}

\sectionfont{\fontsize{10}{15}\selectfont}

\begin{document}
	\maketitle

\noindent Consider a database consisting of the relations, with primary keys in bold:\\
\\
customer(\textbf{cid}, cname, rating, salary)\\
order(\textbf{cid, iid, day,} qty)\\
item(\textbf{iid}, iname, type, price)\\
\\

\section{Find the details (i.e., all attributes) of the customers who have a rating higher than 6 and earn less than \$125,000.}
\[
\sigma_{rating > 6 \  \wedge \ salary < 125000}(customer)
\]

\section{Find the names and types of items which were ordered by some customer named `Bob' who is rated above 5.}
\[
\pi_{iname, type}(\sigma_{cname='Bob' \ \wedge \ rating > 5}(customer) \bowtie item)
\]

\section{Consider the query \(\pi_{iname,type}(item)\). Suppose item has 1000 tuples. Then how many tuples
will the result of the above projection query contain? Explain your answer.}

It depends on the actual instance of \textit{item}. The number of tuples can vary from 1 to 1000.\\
\\
\texttt{(iname type)} is not a key. There is also no uniqueness constraint on \texttt{iname} and \texttt{type}. So it is possible that \texttt{iname} and \texttt{type} are the same for these 1000 tuples. In this case, the resulting projection will remove duplicates and result in 1 tuple. It is also possible that \texttt{iname} and \texttt{type} are distinct from tuple to tuple, so the resulting projection will have 1000 distinct tuples. Therefore, it is possible that some of the \texttt{(iname type)} pairs are duplicates and some of them are distinct so the result can vary from 1 to 1000.


\end{document}