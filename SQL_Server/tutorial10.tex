\documentclass{article}
\usepackage{graphicx}
\graphicspath{ {images/} }
\title{Tutorial 10: An Introduction to SQL Server}
\author{
	Jin, Ziyang\\
	\texttt{\# 34893140}\\
	\texttt{f4a0b}
	\and
	Kim, Joon Hyung\\
	\texttt{\# 35183128}\\
	\texttt{l1m8}
}

\begin{document}
	\maketitle

\section{Deliverable 1}

\begin{verbatim}
Table name:   dbo.purchase
Attributes:   pid, cid, iid, day, qty
Primary key:  pid
Foreign Keys: cid REFERENCES dbo.cutomer, iid REFERENCES dbo.item
\end{verbatim}

\section{Deliverable 2}

Theoretically, the first query, which uses the inner join syntax, will keep only 1 copy of the joining attributes. While the second query, which just specifies the join condition, will keep both copies of the joining attributes. However, as a result of SQL Server, these 2 queries both keep only 1 copy of the joining attributes;

\section{Deliverable 3}

\begin{verbatim}
select c.cname
from   customer c, purchase p, item i
where  p.cid = c.cid and i.iid = p.iid 
       and i.iname like '%Chococlate Frog%';
\end{verbatim}

\section{Deliverable 4}
\begin{verbatim}
update purchase
set    qty = 5
where  pid in (select p.pid
               from customer c, purchase p, item i
               where c.cid = p.cid and p.iid = i.iid
                     and c.cname like 'S. Uper'
                     and i.iname like '%Chocolate Frog%');
\end{verbatim}

\section{Deliverable 5}

\begin{verbatim}
delete from purchase
where qty = (select max(p.qty)
             from   purchase p
             where  p.iid = 180);
\end{verbatim}

\noindent before: \\


\noindent after:


\section{Deliverable 6}
\begin{verbatim}
delete from item
where iid = 180;
\end{verbatim}

Because the item with iid 180 is referenced from purchase table. If we delete 

\section{Deliverable 7}


\section{Deliverable 8}


\section{Deliverable 9}




\end{document}