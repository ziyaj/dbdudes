\documentclass{article}
\usepackage{sectsty}
\title{Tutorial 5: Logical Database Design}
\author{
	Jin, Ziyang\\
	\texttt{\# 34893140}\\
	\texttt{f4a0b}
	\and
	Kim, Joon Hyung\\
	\texttt{\# 35183128}\\
	\texttt{l1m8}
}
\date{February 2018}

\sectionfont{\fontsize{12}{15}\selectfont}

\begin{document}
	\maketitle
	\texttt{StudentInfo(S, N, M, A, C, T, I, L, G)}\\
	\\
	with the following FDs:
	\begin{enumerate}
		\item \texttt{S -> N}
		\item \texttt{C -> T, I}
		\item \texttt{I -> L}
		\item \texttt{S, C, M -> G}
		\item \texttt{S, M -> A}
		\item \texttt{A -> M}
	\end{enumerate}
	
\section{Find all keys and prove that you have found them all.}

Since \texttt{S} and \texttt{C} do not appear in any RHS of the FDs, \\
\texttt{S} and \texttt{C} must be part of every key. \\
\( \texttt{(S C)}^+ \) = \texttt{S C N T I L} so \texttt{(S C)} is not a key.\\
\\
Guess \texttt{(S C M)} and \texttt{(S C A)} are keys. \\
\( \texttt{(S C M)}^+ \) = \texttt{(S C M N T I L G A)} = \texttt{SI} so \texttt{(S C M)} is a superkey.\\
\( \texttt{(S C A)}^+ \) = \texttt{(S C A N T I L G M)} = \texttt{SI} so \texttt{(S C A)} is a superkey.\\
Since \texttt{(S C)} is not a key, so \texttt{(S C M)} and \texttt{(S C A)} are minimal, so they are keys. \\
\\
Consider the maximal set \( \texttt{X}  \subseteq \texttt{(S N M A C T I L G)} \) such that \( \texttt{(S C)} \subset \texttt{X}\) and \( \texttt{A, M} \notin \texttt{X} \). The maximal such \texttt{X} is \texttt{(S N C T I L G)} \\
\( \texttt{X}^+ = \texttt{(S N C T I L G)}^+ = \texttt{(S N C T I L G)} \neq \texttt{SI} \)\\
\\
Therefore, \texttt{(S C M)} and \texttt{(S C A)} are the only keys.\\



\section{Find a minimal cover for this set of FDs.}

\section{Obtain a lossless-join, BCNF decomposition of StudentInfo.}

\section{Obtain a lossless-join, dependency-preserving, 3NF decomposition of StudentInfo by
making use of the BCNF decomposition in Question (f).}

\section{Obtain a lossless-join, dependency-preserving, 3NF decomposition of StudentInfo via
synthesis by making use of your minimal cover in Question (e).}

\section{Comment on the differences, if any, between your answers to Questions (g) and (h).}


\end{document}