\documentclass{article}
\usepackage{sectsty}
\title{Tutorial 5: Logical Database Design}
\author{
	Jin, Ziyang\\
	\texttt{\# 34893140}\\
	\texttt{f4a0b}
	\and
	Kim, Joon Hyung\\
	\texttt{\# 35183128}\\
	\texttt{l1m8}
}
\date{February 2018}

\sectionfont{\fontsize{12}{15}\selectfont}

\begin{document}
	\maketitle
	\texttt{StudentInfo(S, N, M, A, C, T, I, L, G)}\\
	\\
	with the following FDs:
	\begin{enumerate}
		\item \texttt{S -> N}
		\item \texttt{C -> T, I}
		\item \texttt{I -> L}
		\item \texttt{S, C, M -> G}
		\item \texttt{S, M -> A}
		\item \texttt{A -> M}
	\end{enumerate}
	
\section{Find all keys and prove that you have found them all.}

Since \texttt{S} and \texttt{C} do not appear in any RHS of the FDs, \\
\texttt{S} and \texttt{C} must be part of every key. \\
\( \texttt{(S C)}^+ = \texttt{S C N T I L} \neq \texttt{SI} \), so \texttt{(S C)} is not a key.\\
\\
Guess \texttt{(S C M)} and \texttt{(S C A)} are keys. \\
\( \texttt{(S C M)}^+ \) = \texttt{(S C M N T I L G A)} = \texttt{SI} so \texttt{(S C M)} is a superkey.\\
\( \texttt{(S C A)}^+ \) = \texttt{(S C A N T I L G M)} = \texttt{SI} so \texttt{(S C A)} is a superkey.\\
Since \texttt{(S C)} is not a key, so \texttt{(S C M)} and \texttt{(S C A)} are minimal, so they are keys. \\
\\
Consider the maximal set \( \texttt{X}  \subseteq \texttt{(S N M A C T I L G)} \) such that \( \texttt{(S C)} \subset \texttt{X}\) and \( \texttt{A, M} \notin \texttt{X} \). The maximal such \texttt{X} is \texttt{(S N C T I L G)} \\
\( \texttt{X}^+ = \texttt{(S N C T I L G)}^+ = \texttt{S N C T I L G} \neq \texttt{SI} \)\\
So any set other than \texttt{(S C M)} and \texttt{(S C A)} cannot be a key.\\
\\
Therefore, \texttt{(S C M)} and \texttt{(S C A)} are the only keys for the relation \texttt{SI}.\\


\section{Find a minimal cover for this set of FDs.}

\begin{enumerate}
\item Put the FDs in a Standard Form.\\
	\texttt{S->N, C->T, C->I, I->L, SCM->G, SM->A, A->M}.
\item Minimize the Left Side of Each FD.\\
	Single attribute left sides are already minimized.\\
	\texttt{SCM->G}: \texttt{S,C} do not appear on LHS. \texttt{M} cannot be reduced from \texttt{S,C}.\\
	\texttt{SM->A}: \texttt{S} does not appear on LHS. \texttt{M} cannot be reduced from \texttt{S}.\\
	 So the left sides are all minimized.
\item Delete Redundant FDs.\\
	\texttt{N, T, I, L, G, A, M} only appear once on RHS, so there is no redundant FD.
\end{enumerate}
So the minimum cover for this set of FDs is:
	\begin{enumerate}
		\item \texttt{S -> N}
		\item \texttt{C -> T}
		\item \texttt{C -> I}
		\item \texttt{I -> L}
		\item \texttt{S, C, M -> G}
		\item \texttt{S, M -> A}
		\item \texttt{A -> M}
	\end{enumerate}

\section{Obtain a lossless-join, BCNF decomposition of StudentInfo.}

For FD (1): \texttt{S->N}, we decompose \texttt{SI} to\\
\texttt{S1 = (S N)} \\
\texttt{S2 = (S M A C T I L G)} \\
\\
For FD (3): \texttt{I->L}, we decompose \texttt{S2} to\\
\texttt{S3 = (I L)} \\
\texttt{S4 = (S M A C T I G)} \\
\\
For FD (2): \texttt{C->T,I}, we decompose \texttt{S4} to \\
\texttt{S5 = (C T I)} \\
\texttt{S6 = (S M A C G)} \\
\\
For FD (4): \texttt{S,C,M->G}, we decompose \texttt{S6} to \\
\texttt{S7 = (S C M G)} \\
\texttt{S8 = (S C M A)} \\
\\
For FD (6): \texttt{A->M}, we decompose \texttt{S8} to \\
\texttt{S9 = (A M)} \\
\texttt{S10 = (S C A)}\\
\\
Note that FD (5) cannot be applied to any final decomposed relation, so the decomposition is done. FD (5): \texttt{S,M->A} is not preserved. So the final decomposition of StudentInfo is:\\
\texttt{S1 = (S N)} \\
\texttt{S3 = (I L)} \\
\texttt{S5 = (C T I)} \\
\texttt{S7 = (S C M G)} \\
\texttt{S9 = (A M)} \\
\texttt{S10 = (S C A)}\\



\section{Obtain a lossless-join, dependency-preserving, 3NF decomposition of StudentInfo by
making use of the BCNF decomposition in Question (c).}
The BCNF decomposition we obtained from (c) does not preserve FD (5): \texttt{S,M->A}, so we add \texttt{S11 = (S M A)}.\\
So the final 3NF decomposition is: \\
\texttt{S1 = (S N)} \\
\texttt{S3 = (I L)} \\
\texttt{S5 = (C T I)} \\
\texttt{S7 = (S C M G)} \\
\texttt{S9 = (A M)} \\
\texttt{S10 = (S C A)}\\
\texttt{S11 = (S M A)}

\section{Obtain a lossless-join, dependency-preserving, 3NF decomposition of StudentInfo via
synthesis by making use of your minimal cover in Question (b).}
The minimum cover is \texttt{S->N, C->T, C->I, I->L, SCM->G, SM->A, A-M}.\\
The relations obtained from a minimum cover set is:\\
\texttt{R1 = (S N)}\\
\texttt{R2 = (C T)}\\
\texttt{R3 = (C I)}\\
\texttt{R4 = (I L)}\\
\texttt{R5 = (S C M G)}\\
\texttt{R6 = (S M A)}\\
\texttt{R7 = (A M)}\\



\section{Comment on the differences, if any, between your answers to Questions (d) and (e).}


\end{document}