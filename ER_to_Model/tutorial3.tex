\documentclass{article}
\title{%
	Tutorial 3: Logical Database Design \\
	\large Mapping ER Diagrams to the Relational Model}
\author{
	Jin, Ziyang\\
	\texttt{\# 34893140}\\
	\texttt{f4a0b}
	\and
	Kim, Joon Hyung\\
	\texttt{\# 35183128}\\
	\texttt{l1m8}
}
\date{January 2018}

\begin{document}
	\maketitle
	\begin{enumerate}
		\item \underline{1:1} \\
			We choose to merge the relation to X side:\\
			X\_R(\underline{K1}, A1, \textbf{K2}) \\
			\texttt{K2 REFERENCES Y}\\
			\\
			Y(\underline{K2}, A2) \\
			
		\item \underline{1:M} \\
			\\
			X(\underline{K1}, A1) \\
			\\
			Y\_R(\underline{K2}, A2, \textbf{K1}) \\
			\texttt{K1 REFERENCES X}\\
			
		\item \underline{M:N(Binary Relationship)} \\
			\\
			X(\underline{K1}, A1) \\
			\\
			Y(\underline{K2}, A2) \\
			\\
			R(\textbf{\underline{K1, K2,}} A3) \\
			\texttt{K1 REFERENCES X,}\\
			\texttt{K2 REFERENCES Y} \\
			
		\item \underline{M:N(Ternary Relationship)} \\
			\\
			X(\underline{K1}, A1) \\
			\\s
			Y(\underline{K2}, A2) \\
			\\
			Z(\underline{K3}, A3) \\
			\\
			R(\textbf{\underline{K1, K2, K3,}} A4) \\
			\texttt{K1 REFERENCES X,}\\
			\texttt{K2 REFERENCES Y,}\\
			\texttt{K3 REFERENCES Z} \\
			
		\item \underline{1:M Strong Entity with Total Participation}\\
			\\
			X(\underline{K1}, A1) \\
			\\
			Y\_R(\underline{K2}, A2, \textbf{K1}) \\
			\texttt{K1 REFERENCES X, K1 cannot be null}\\
			
		\item \underline{1:M Weak Entity with Total Participation (assume that A2 is the partial key)} \\
			\\
			X(\underline{K1}, A1) \\
			\\
			Y(\underline{A2, \textbf{K1},} A3) \\
			\texttt{K1 REFERENCES X, ON DELETE CASCADE, ON UPDATE CASCADE} \\
			
		\item \underline{1:1 and 1:M Unary Relationship} \\
			\\
			X(\underline{K1}, A1, \textbf{RK1}) \\
			\texttt{RK1 REFERENCES X(K1)}\\
			
		\item \underline{M:N Unary Relationship} \\
			\\
			X(\underline{K1}, A1) \\
			\\
			R(\textbf{\underline{R1K1, R2K1}}) \\
			\texttt{R1K1 REFERENCES X(K1),}\\
			\texttt{R2K1 REFERENCES X(K1)}\\
			
		\item \underline{ISA 1}\\
			Assumption: not disjoint and not covering\\
			\\
			X(\underline{K1}, A1) \\
			\\
			Y(\textbf{\underline{K1}}, A2, A3)\\
			\texttt{K1 REFERENCES X} \\
			\\
			Z(\textbf{\underline{K1}}, A4, A5)\\
			\texttt{K1 REFERENCES X} \\
			
		\item \underline{ISA 2 (the "d"�� means disjoint)} \\
			Assumption: disjoint and covering \\
			\\
			Y(\underline{K1}, A1, A2, A3) \\
			\\
			Z(\underline{K1}, A1, A4, A5) \\
	\end{enumerate}
	
\end{document}
