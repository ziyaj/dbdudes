\documentclass{article}
\title{%
	Tutorial 3: Logical Database Design \\
	\large Mapping ER Diagrams to the Relational Model}
\author{
	Jin, Ziyang\\
	\texttt{\# 34893140}\\
	\texttt{f4a0b}
	\and
	Kim, Joon Hyung\\
	\texttt{\# 35183128}\\
	\texttt{l1m8}
}
\date{January 2018}

\begin{document}
	\maketitle
	\begin{enumerate}
		\item \underline{1:1} \\
			X(\underline{K1}, A1, \textbf{K2}) \\
			Y(\underline{K2}, A2) \\
		\item \underline{1:M} \\
			X(\underline{K1}, A1) \\
			Y(\underline{K2}, A2, \textbf{K1}) \\
		\item \underline{M:N(Binary Relationship)} \\
			X(\underline{K1}, A1) \\
			Y(\underline{K2}, A2) \\
			R(\textbf{\underline{K1, K2,}} A3) \\
		\item \underline{M:N(Ternary Relationship)} \\
			X(\underline{K1}, A1) \\
			Y(\underline{K2}, A2) \\
			Z(\underline{K3}, A3) \\
			R(\textbf{\underline{K1, K2, K3,}} A4) \\
		\item \underline{1:M Strong Entity with Total Participation}\\
			X(\underline{K1}, A1) \\
			Y(\underline{K2}, A2, \textbf{K1}) \\
		\item \underline{1:M Weak Entity with Total Participation (assume that A2 is the partial key)} \\
			X(\underline{K1}, A1) \\
			Y(\underline{\textbf{K1}, A2}, A3) \\
		\item \underline{1:1 and 1:M Unary Relationship} \\
			X(\underline{K1}, A1, \textbf{RefK1}) \\
		\item \underline{M:N Unary Relationship} \\
		\item \underline{ISA 1}\\
			X(\underline{K1}, A1) \\
			Y(\textbf{\underline{K1}}, A2, A3) \\
			Z(\textbf{\underline{K1}}, A4, A5) \\
		\item \underline{ISA 2 (the “d” means disjoint)} \\
			X(\underline{K1}, A1) \\
			Y(\textbf{\underline{K1}}, A2, A3) \\
			Z(\textbf{\underline{K1}}, A4, A5) \\
	\end{enumerate}
	
\end{document}
